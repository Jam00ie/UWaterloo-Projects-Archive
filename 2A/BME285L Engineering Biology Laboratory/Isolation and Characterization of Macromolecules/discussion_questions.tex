\section*{Discussion Questions}

\textbf{Why is it important to maintain a constant volume of water in your water bath (when boiling nucleic acid fractions)?}\par \medskip
If the water bath is not maintained at constant volume, it will eventually evaporate to the point where the nucleic acid fractions are no longer submurged in the water bath. This can cause uneven heating of the samples which would lead to inaccurate results.
\par \bigskip \noindent
\textbf{Why are test tubes with red caps incubated at 37 °C?\@ Why are the test tubes kept in the refrigerator until they are to be used for chromatography?}\par \medskip
The pancreatic enzymes added to the test tubes with red caps are incubated at 37 °C to simulate body tempurature which the enzymes are active in. During incubation, the enzymes hydrolyzes the protein molecules into amino acid subunits. Once the tubes are finished incubating for 24 hours, they are refrigerated until chromatography to prevent contamination and keep the hydrolysed contents of the tube stable.
\par \bigskip \noindent
\textbf{If all of the isolation and fractionation experiments were 100\% successful, what substances or molecules would you expect to be present in:}
    \begin{enumerate}[label = (\alph*)]
        \item \textbf{The tube labelled ``hydrolyzed protein''}\par Amino acids, such as lysine, leucine, aspartic acid, glutamic acid, and isoleucine which are major components of yeast protein~\citep{ABDELHAFEZ1977631}.
        \item \textbf{The tube labelled ``unhydrolyzed protein''}\par As no hydrolysis of protein took place, there would be no amino acids in the tube, only whole protein molecules.
        \item \textbf{The tube labelled ``hydrolyzed nucleic acids''}\par As the amount of RNA is much greater than the amount of DNA in yeast cells~\citep{OLM2023}, the hydrolyzed nuclic acid will contain RNA base subunits uracil, cytosine, adenine and guanine as well as the sugar-phosphate subunits. Guanine, once hydrolyzed from the nucleic acid, will no longer remain in solution~\citep{OLM2023}.
        \item \textbf{The tube labelled ``unhydrolyzed nucleic acids''}\par As no hydrolysis of nucleic acids took place, there would be no base subunits in the tube, only whole nucleic acid polymers.
    \end{enumerate}
\noindent
\textbf{If the hydrolysis steps carried out last week were only partially successful, what substances or molecules would you expect to be present in:}
    \begin{enumerate}[label = (\alph*)]
        \item \textbf{The tube labelled ``hydrolyzed protein''}\par A mixture of proteins and amino acids found in yeast cells would be present.
        \item \textbf{The tube labelled ``hydrolyzed nucleic acids''}\par A mixture of partially broken down nucleic acid polymers, nucleotide molecules, and individual base and sugar-phosphate subunits would be present.
    \end{enumerate}
\noindent
\textbf{Into what category of substances do alanine, histidine, aspartic acid, lysine, and methionine fall? What about adenine, cytosine, and uracil?}\par
\indent
Alanine, histidine, aspartic acid, lysine and methoinine are amino acids. Adenine, cytosine and uracil are nitrogenous bases of nucleic acids.
