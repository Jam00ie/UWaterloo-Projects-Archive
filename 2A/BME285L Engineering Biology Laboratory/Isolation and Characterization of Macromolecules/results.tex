\section*{Results}
    \subsection*{Sample Calculation}
    \[
        R_{f_{\text{alanine}}}
        = \frac{\text{Distance (from origin) travelled by substance (cm)}}{\text{Distance (from origin) travelled by solvent (cm)}}
        = \frac{9.1 \text{cm}}{11.9 \text{cm}}
        = 0.76
    \]
    \subsection*{Protein Chromatogram}
    \begin{align*}
        R_{f_{\text{unhydrolyzed protein (dark purple smudge)}}}\
        &=\ 0.57\\
        R_{f_{\text{unhydrolyzed protein (pink smudge)}}}\
        &=\ 0.71\\
        R_{f_{\text{hydrolyzed protein (dark purple smudge)}}}\
        &=\ 0.57\\
        R_{f_{\text{hydrolyzed protein (light purple smudge)}}}\
        &=\ 0.76\\
        R_{f_{\text{hydrolyzed protein (very light purple smudge)}}}\
        &=\ 0.94\\
        R_{f_{\text{alanine}}}\
        &=\ 0.76\\
        R_{f_{\text{histidine}}}\
        &=\ 0.46\\
        R_{f_{\text{aspartic acid}}}\
        &=\ 0.56\\
        R_{f_{\text{lysine}}}\
        &=\ 0.57\\
        R_{f_{\text{methionine}}}\
        &=\ 0.88\\
        R_{f_{\text{unknown amino acid}}}\
        &=\ 0.57
    \end{align*}
    \subsection*{Protein Chromatogram --- Explanation}
    In the chromatogram for hydrolyzed protein, there are three to four distinguishable smudges from the ninhydrin. The dark purple smudge in the center has an RF value of 0.57, which matches that of lysine. The purple smudge above it has an RF value that matches one of alanine. Hints of pink can be seen hidden by the dark purple smudge; as ninhydrin produces different types of purple depending on the chemical nature of the amino acid it reacts with~\citep{PERRETT2014598}, the hidden pink smudge can be matched to aspartic acid in shade and in RF value. The faint purple smudge at the very top may represent methionine which its RF value is most similar to, as methionine does exist in small amounts in yeast protein~\citep{cjf-201606-0012}. The unknown amino acid is most likely lysine as it has an RF value of 0.57 which matches the lysine sample given. \par
    It is unexpected that the unhydrolyzed protein also produced visible smudges, as no amino acids should have been present to react with ninhydrin. This may have been caused by contamination during the lab process, introducing amino acids into the sample and creating visible smudges.
    \subsection*{Nucleic Acid Chromatogram}
    \begin{align*}
        R_{f_{\text{unhydrolyzed nucleic acid}}}\
        &=\ \text{N/A}\\
        R_{f_{\text{hydrolyzed nucleic acid (furthest smudge)}}}\
        &=\ 0.65\\
        R_{f_{\text{hydrolyzed nucleic acid (middle smudge)}}}\
        &=\ 0.43\\
        R_{f_{\text{hydrolyzed nucleic acid (closest smudge)}}}\
        &=\ 0.23\\
        R_{f_{\text{adenine}}}\
        &=\ 0.65\\
        R_{f_{\text{cytosine}}}\
        &=\ 0.51\\
        R_{f_{\text{uracil}}}\
        &=\ 0.57\\
        R_{f_{\text{adenine-cytosine-uracil (furthest smudge)}}}\
        &=\ 0.64\\
        R_{f_{\text{adenine-cytosine-uracil (closest smudge)}}}\
        &=\ 0.47
    \end{align*}
    \subsection*{Nucleic Acid Chromatogram --- Explanation}
    In the chromatogram for hydrolyzed nucleic acid, there are three spots marked after UV light visualization. The spot farthest from the initial line matches the RF value of adenine most. The other two spots of the hydrolyzed nucleic acid does not seem to match the RF values of the given samples. This is most likely from contamination during the lab process from excess supernatents, or salts. \par
    The unhydrolyzed nucleic acid produced no visible spots as expected; the adenine-cytosine-uracil sample seem to match the individual base samples, with uracil not being marked due to close proximity with adenine.