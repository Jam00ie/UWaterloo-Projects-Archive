\section*{Summary}
This experiment's objective was to isolate and characterize macromolecules from yeast cells, a model organism used to study human physiology, development and diseases.
Important techniques used during the lab include centrifugation; used to separate pellets from supernatants in solutions, and chromatography; used to distinguish individual compounds from the protein and nucleic acid mixture samples.
In the chromatography results for hydrolyzed protein, distinct smudges indicated presence of lysine, alanine, and aspartic acid in the sample. 
In the results for hydrolyzed nucleic acid, adenine was detected; however, no other matches to other base subunits were found, indicating unsuccessful conduction of the procedures.
Overall, this lab demonstrated the techniques to find the composition of macromolecules in yeast cells, with some unexpected results that emphasized the importance of careful lab techniques for accuracy.