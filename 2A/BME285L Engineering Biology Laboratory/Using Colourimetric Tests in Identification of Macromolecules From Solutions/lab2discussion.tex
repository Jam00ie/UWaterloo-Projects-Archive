\section*{Discussion}

Under iodine testing, 1\% glycogen solution produced a red-brown colouration indicating presence of glycogen, and 1\% starch solution produced a blue-black colouration indicating presence of starch.

Under Benedect's test, 1\% glucose, 1\% maltose, 1\% lactose, and unknown solution \#39 produced a brick-red precipitate, indicating high concentrations of reducing sugars. 5\% honey produced a brown colour, 1\% glycogen produced a minor amount of orange precipitate, and beer produced orange precipitate; all indicating some concentration of reducing sugar in solution.

Under Biuret testing, only 1\% protein produced a violet colouration indicating presence of protein.

As expected, all the postive controls produced a positive result in their respective tests. Most sample solutions with known chemical species behaved accordingly; however, there were some unexpected results. Notable observations will be discussed.

The Biuret test did not indicate the presence of protein in the 1\% glycogen solution, which was unexpected as glycogen contains a protein called glycogenin in its core. It may have been that the thick branching of glucose chains surrounding glycogenin prevented the interaction of the Biuret reagent with the protein.

1\% glycogen testing slightly positive under Benedict's test was also unexpected; although glycogen has one reducing end per molecule, it is covalently bonded to glycogenin, preventing the redox reaction in theory. It may have been that during the boiling process, some hydrolysis may have taken place that created reducing sugars.

The unknown solution \#39 tested positive to Benedict's test only; the brick-red colouration indicates that this solution contains reducing sugars, such as glucose, maltose, and lactose.


