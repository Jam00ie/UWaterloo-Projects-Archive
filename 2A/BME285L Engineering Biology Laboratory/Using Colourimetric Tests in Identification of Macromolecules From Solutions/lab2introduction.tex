\section*{Introduction}

% Info about macromolecules and quick summary of lab
    Carbohydrates, lipids, proteins and nucleic acids are the major classes of biological macromolecules that make up biological systems.
    In this experiment, colourimetric tests were used to identify some types of macromolecules present in a variety of sample compound solutions.
    The macromolecules identified in this experiment include protein, carbohydrates such as starch and glycogen, as well as reducing sugars which are present in carbohydrate macromolecules. \par

% Iodine Test
    % Structure of starch, amylose, amylopectin, and glycogen
    Starch is used by plants as a storage for carbohydrates. 
    It is composed of two different structures of glucose polymers; amylase and amylopectin.
    Amylose exhibits a helical coiled structure, primarily attributed to its α-1,4-glycosidic linkages connecting glucose monomers. 
    In contrast, amylopectin, while also featuring α-1,4-glycosidic linkages, incorporates α-1,6-glycosidic linkages that introduce branching within the polymer. 
    This branching pattern, occurring approximately every 25 to 30 glucose units, disrupts the helical coils seen in amylose.
    Interestingly, glycogen, a molecule employed by animals for energy storage, closely resembles amylopectin in structure of the glucose polyer that surrounds glycogenin.
    However, glycogen displays a higher density of α-1,6-glycosidic linkages, occurring at intervals of roughly 8 to 14 glucose units. 
    This increased frequency of branching serves to further reduce the prevalence of helical coils within the glycogen structure. \par

    % lugol's iodine & formation of amylose-iodine complex
    The unique helical structures present in the previously mentioned polysaccharides are crucial in the mechanism of iodine tests, one of the colourimetric tests used in this experiment.
    A key material used in the iodine testing process is Lugol's iodine, which is a solution of 5\% elemental iodine, 10\% potassium iodide and distilled water.
    Elemental iodine I\textsubscript{2} is non-polar and thus has limited solubility in water. 
    However, when potassium iodide salts are added to the solution, I\textsubscript{2} reacts with the dissociated iodine ions I\textsuperscript{-} to form polyiodide ions such as I\textsubscript{3}\textsuperscript{-} which are soluble in water \parencite{Calabrese2000}.
    The helical structures provide space within their cores for the polyiodide ions to fit, resulting in the formation of amylose-iodine complexes, which belong to a category known as charge transfer (CT) complexes \parencite{Goedecke2016}. \par

    % Ct complexes, wavelengths, and resulting colours
    CT complexes are formed when two or more molecules are stabilized by electrostatic attraction, with one molecule acting as an electron donor and the other as an electron acceptor \parencite{Aly2014}.
    During the electronic transition from donor to acceptor molecules, light is absorbed, producing intense color visible to the human eye corresponding to the complementary shade of the absorbed wavelength \parencite{Calatayud2013, Goedecke2016}.
    In the case of the amylose-iodine complex, the length of the helical coils determine the colour produced; longer coils undistrupted by braching absorb more wavelengths of light \parencite{Brust2020}. 
    Thus, amylose gives a deep blue-violet colour due to its lack of branching while amylopectin and glycogen gives a reddish brown colouration \parencite{Ball2011}. 
    While starch contains more amylopectin than amylose, the colour produced from amylose is far more intense; the colour produced from amylopectin is mostly overshadowed and results in the signature blue-black colouration. \par

    % positive controls
    In this experiment's iodine test, the 1\% glycogen solution and the 1\% starch solution was used as positive controls, with each solution expected to produce reddish brown and blue-black colouration respectively. \par

% Benedict's Test
    Another colourimetric test used in this experiment is Benedict's test, which uses alkaline copper (II)-citrate complex (Benedict's reagent) to identify reducing sugars in solution \parencite{Markina2016}.
    Reducing sugars contain hemiacetal/hemiketal groups (aldehyde/ketone in straight-chain form) that act as a reducing agent.
    Redox reactions between the reducing agent of sugars and the cuprous ion from the Benedict's reagent results in the formation of Cu\textsubscript{2}O particles, which vary in resulting size depending on the reaction conditions such as pH \parencite{Markina2016}. 

    The Cu\textsubscript{2}O particles absorb different wavelengths of light depending on its size; bigger Cu\textsubscript{2}O particles produced in reaction lead to brick-red colouration; smaller particles lead to yellow colouration \parencite{Markina2016}.
    Due to the blue colour of the Benedict's reagent, positive results of the Bendict's test will display a variety of hues; small concentrations of smaller particles will produce green colouration due to yellow and blue mixing, while large concentrations of bigger particles will be brick-red.

    In the samples provided, 1\& glucose, 1\% maltose, and 1\% lactose solutions will be used as the positve control for Benedict's test as they are solutions of reducing sugars.

% Biuret test

    The final test utilized in this experiment is the Biuret test, which identify the prescence of peptide bonds, and in turn proteins. Cu\textsuperscript{2+} ions in solution are reduced by peptide bonds in alkaline conditions, and the reaction forms a coordination complex that absorbs light; producing a violet colour \parencite{Bhagavan2002}. The intensity of the colouration is determined by the concentration of coordination complexes formed. 1\% copper sulfate and sodium hydroxide solutions were used in this experiment; and the 1\% protein sample was used as the postive control.

    In all three colourimetric tests, distilled water was used as the negative control; pure H\textsubscript{2}O does not react and produce a colour with any chemicals used in this experiment for identification.


    